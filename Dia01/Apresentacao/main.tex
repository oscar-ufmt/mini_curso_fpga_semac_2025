\documentclass[aspectratio=169]{beamer}
\usetheme{Boadilla}



\usepackage[utf8]{inputenc}       % Codificação de entrada
\usepackage[T1]{fontenc}          % Codificação de saída
\usepackage[portuguese]{babel}   % Idioma
\usepackage{XCharter}             % Fonte principal
\AtBeginDocument{\fontsize{12}{12}\selectfont}

%\usepackage{geometry}
%\geometry{papersize={210mm,150mm}}

\usepackage{color}
\usepackage{graphicx}
\usepackage{amsmath, amssymb, amsfonts}
\usepackage{bm}
\usepackage{booktabs}
\usepackage{multirow}
\usepackage{caption}
\usepackage{subfigure}
\usepackage{wrapfig}
\usepackage{multicol}
\usepackage{sidecap}
\usepackage{tikz}
\usepackage{listings}
\usepackage{siunitx}
\usepackage{pdfpages}
\usepackage{float}
\usepackage{hyperref}
\usepackage{academicons}
\usepackage{setspace}

\newcommand{\eng}[1]{\textsl{#1}}
\newcommand{\cod}[1]{\texttt{#1}}
\title[Minicurso FPGAs]{\huge   Introdução ao Projeto Digital Avançado com VHDL e FPGA} 


\author[Prof. Dr. Oscar Eduardo Anacona Mosquera]{Prof. Dr. Oscar Eduardo Anacona Mosquera \newline\newline 
\scriptsize{oscar.mosquera@ufmt.br}
}


\AtBeginSection[]
{
	\begin{frame}{Conteúdo}
		\tableofcontents[currentsection]
	\end{frame}
}


\begin{document}

%%====================================================================================
\begin{frame}[plain]
\titlepage
%\vspace{3cm}
%\begin{center} \color{MULTurquoise} {Chair of Cyber-Physical-Systems}\end{center}

\end{frame}
%%====================================================================================

\section{Agradecimentos}


%%====================================================================================
\begin{frame}{Agradecimentos}
	
	\begin{itemize}
		\item Universidade Federal de Mato Grosso (UFMT)
		\item Faculdade de Engenharia do Campus de Várzea Grande (FAENG)
		\item Instituto de Computação (IC-UFMT)
		\item Curso da Engenharia da Computação
		\item Prof. Dr. Carlos Humberto Llanos (QEPD), Professor Associado do Departamento de Engenharia Mecânica (ENM) da Universidade de Brasília (UnB)
		\item  Intel Altera
		%\item  O técnico Igor
		\item  Colaboradores
	\end{itemize}	
	
\end{frame}
%%====================================================================================
\section{Objetivos}

%%====================================================================================
\begin{frame}
	\frametitle{Objetivos}
	
	\begin{itemize}
		\item \textbf{VHDL e Projeto Digital:}
		\begin{itemize}
			\item Introduzir os conceitos fundamentais do projeto digital utilizando a linguagem VHDL.
		\end{itemize}
		
		\item \textbf{Simulação e Implementação em FPGA:}
		\begin{itemize}
			\item Capacitar os participantes na simulação e verificação de projetos digitais utilizando testbenches.
		\end{itemize}
		
		\item \textbf{Testbench com ModelSim:}
		\begin{itemize}
			\item Apresentar o software ModelSim para simulação de circuitos digitais.
			\item Demonstrar como criar um testbench para verificar o comportamento do circuito antes da implementação em hardware.
		\end{itemize}		
		
		
		\item \textbf{Projeto Aplicado – Carrinho Seguidor de Linha:}
		\begin{itemize}
			\item Aplicar os conhecimentos adquiridos no desenvolvimento de um projeto prático e funcional.
			\item Controlar um carrinho seguidor de linha utilizando lógica implementada em FPGA.
		\end{itemize}
		

		
		\item \textbf{Comparação com Arduino Uno:}
		\begin{itemize}
			\item Apresentar uma implementação alternativa do projeto utilizando a plataforma Arduino Uno.
		\end{itemize}
	\end{itemize}


\end{frame}
%%====================================================================================


\section{Apresentação}
%%====================================================================================
\begin{frame}
	
	\frametitle{Apresentação}
	\begin{itemize}
		\justifying
		%\setbeamertemplate{itemize item}[triangle]
		\item \textbf{Nome:} Prof. Dr. Oscar Eduardo Anacona Mosquera
		\item \textbf{Formação:} Doutor e Mestre em Sistemas Mecatrônicos pela Universidade de Brasília (UnB), Bacharel em Engenharia Física pela Universidad del Cauca (Colômbia) e revalidado pela Universidade de Goiás (UFG).
		\item \textbf{Áreas de Atuação:} Sistemas embarcados, Automação, Robótica, e Algoritmos de otimização.
		\begin{itemize}
			\item FPGAs, microcontroladores (PIC e Atmega) e CLPs.
		\end{itemize}
		
		\item \textbf{Professor da UFMT do curso da Engenharia da Computação da FAENG-VG.}

		\item \textbf{GitHub:} oscar-ufmt
		\item \textbf{CV Lattes:} \url{https://lattes.cnpq.br/4776138897349156}
	\end{itemize}
\end{frame}
%%====================================================================================

\section{Ementa do minicurso}

%%====================================================================================
\begin{frame}
	\frametitle{Ementa do minicurso}
	
	
\begin{itemize}
	\item \textbf{Dia 1 – Fundamentos e Ferramentas:}
	\begin{itemize}
		\item Introdução ao conceito de hardware reconfigurável e aplicações em sistemas digitais.
		\item Conceitos básicos da linguagem VHDL: sintaxe, estruturas e modelagem de circuitos simples.
		\item Tutorial prático do Quartus II: criação de projetos, entrada de código, compilação e programação da FPGA.
		\item Tutorial do ModelSim: simulação de circuitos digitais e criação de testbenches para validação funcional.
	\end{itemize}
	
	\item \textbf{Dia 2 – Aplicações Práticas e Comparações:}
	\begin{itemize}
		\item Desenvolvimento de máquinas de estados finitas (FSM) no Quartus, com foco no controle sequencial.
		\item Elaboração de testbenches para validação de FSMs e controle de sistemas digitais.
		\item Implementação do controle de um carrinho seguidor de linha usando VHDL em FPGA (DE0-Nano).
		\item Implementação equivalente do controle do carrinho utilizando Arduino Uno para fins de comparação entre as plataformas.
	\end{itemize}
\end{itemize}

	
\end{frame}
%%====================================================================================

\section{Produção bibliográfica}


%%====================================================================================
\begin{frame}{Produção bibliográfica}
	
	
	\begin{itemize}
	\justifying
	
	\item CABRAL, FELIPE ; ANACONA-MOSQUERA, OSCAR ; SAMPAIO, RENATO C. ; TEODORO, GEORGE ; LLANOS, CARLOS H. ; JACOBI, RICARDO P. . Optimized execution of morphological reconstruction in large medical images on embedded devices. Journal of Real-Time Image Processing, v. 1, p. 1, 2020. 			
	
	\item ANACONA-MOSQUERA, OSCAR; SANTOS, CARLOS E. ; CABRAL, FELIPE R. G. ; SAMPAIO, RENATO C. ; TEODORO, GEORGE ; JACOBI, RICARDO P. ; LLANOS, CARLOS H. . Hardware-Based Fast Hybrid Morphological Reconstruction. IEEE Design \& Test, v. 1, p. 1-1, 2019. 
	
%	\item ANACONA-MOSQUERA, OSCAR; CABRAL, FELIPE R. G. ; SAMPAIO, RENATO C. ; TEODORO, GEORGE ; JACOBI, RICARDO P. ; LLANOS, CARLOS H. . Efficient Hardware Implementation of the Fast Hybrid Morphological Reconstruction Algorithm. In: 2018 31st Symposium on Integrated Circuits and Systems Design (SBCCI), 2018, Bento Gonçalves - RS. 2018 31st Symposium on Integrated Circuits and Systems Design (SBCCI), 2018. p. 1. 
	
%	\item ANACONA-MOSQUERA, OSCAR; TEODORO, GEORGE ; VINHAL, GUSTAVO ; JACOBI, RICARDO P. ; SAMPAIO, RENATO C. ; LLANOS, CARLOS H. . Efficient hardware implementation of morphological reconstruction based on sequential reconstruction algorithm. In: the 30th Symposium, 2017, Fortaleza. Proceedings of the 30th Symposium on Integrated Circuits and Systems Design Chip on the Sands - SBCCI '17. New York: ACM Press, 2017. p. 162. 
	
%	\item ANACONA-MOSQUERA, OSCAR; ARIAS-GARCIA, JANIER ; MUNOZ, DANIEL M. ; LLANOS, CARLOS H. . Efficient hardware implementation of the Richardson-Lucy Algorithm for restoring motion-blurred image on reconfigurable digital system. In: 2016 29th Symposium on Integrated Circuits and Systems Design (SBCCI), 2016, Belo Horizonte. 2016 29th Symposium on Integrated Circuits and Systems Design (SBCCI). p. 1. 
	
	\end{itemize}
	
\end{frame}
%%====================================================================================



\end{document}
